\documentclass[a0paper]{tikzposter}

\usepackage{mathptmx}       % Mathematical PostScript fonts
\usepackage{array}          % Mathematical PostScript fonts
\usepackage{multicol}       % Mathematical PostScript fonts

\definebackgroundstyle{dlsbackground}{
    \node [anchor=south west, inner sep=0, line width=0] at (bottomleft)
        {\includegraphics{diamond-background.png}};
    \node [anchor=south east, inner sep=0, line width=0,
           xshift=-1.25cm, yshift=1.16cm]
        at (bottomleft -| topright)
        {\includegraphics[scale=1.52]{diamond_logo.eps}};
}

\title{\begin{minipage}{\linewidth}\centering
    Architecture of Transverse MultiBunch Feedback Processor at Diamond
    \vspace{1cm}
\end{minipage}
}

\author{M.G.~Abbott, G.~Rehm, I.S.~Uzun}
\institute{Diamond Light Source, Oxfordshire, UK}

\hyphenpenalty=5000
\setlength{\columnsep}{2cm}

\usetheme{Envelope}
\usebackgroundstyle{dlsbackground}

\begin{document}

\maketitle

\begin{columns}
    \column{0.5}
    \block{TMBF Overview}{
        \begin{center}
            \includegraphics[scale=4.25, page=1]{figures.pdf}
        \end{center}
        The Transverse Multi-Bunch Feedback system measures the position of each
        bunch, detects the betatron oscillations of each bunch, and generates a
        drive signal to suppress the oscillations.
    }

    \column{0.5}
    \block{Libera System Platform}{
        \begin{center}
            \includegraphics[scale=3, page=2]{figures.pdf}
        \end{center}
        The DLS TMBF system is implemented on the Instrumentation Technologies
        Libera platform with the control system running EPICS on an ARM based
        embedded Single Board Controller (SBC).
    }
\end{columns}

\block{FPGA System Design And Data Paths}{
    \begin{tabular}{c m{5cm}}
        \begin{minipage}{0.55\textwidth}
            \includegraphics[scale=2.75, page=5]{figures.pdf}
        \end{minipage}
    &
        \begin{minipage}{0.35\textwidth}
        The main function of TMBF is to stabilise transverse oscillations of the
        beam. This is done by running a separate 10-tap FIR on the position of
        each of the stored bunches.  The core data processing chain combines
        this feedback with up to two optional Numerically Controlled Oscillator
        (NCO) outputs.
        \vspace{2cm}
        \newline
        The main data flow is from the ADC, through the FIR with a separate FIR
        filter selected for each bunch, and out through the DAC with the option
        of adding up to two internally generated sine waves. The other paths are
        for control and data capture. The SBC interface controls and
        communicates with all other components of the system: the EPICS
        interface is through this component.
        \end{minipage}
    \end{tabular}
}

\block{Data Processing Chain}{
    \includegraphics[scale=3, page=3]{figures.pdf}
    \includegraphics[scale=3, page=4]{figures.pdf}
    \break
    \begin{multicols}{2}
    Data processing starts by adding a DC offset to each of the four ADC
    channels to compensate for static ADC errors, followed by a 3-tap filter to
    compensate for high frequency phase errors in the front end. The minimum and
    maximum value per bunch of both the ADC and DAC streams is captured for
    display. A 10-tap filter with programmable gain (in 6 dB steps) is applied
    in turn to each bunch in the ring. The output multiplexer adds any
    combination of its three inputs, which is then scaled by a bunch specific
    gain. Finally an output pre-emphasis filter corrects for amplifier errors
    and is followed by a delay line to correctly close the loop.
    \end{multicols}
}

\begin{columns}
\column{0.5}
    \block{Tune PLL}{
    \includegraphics[scale=3, page=6]{figures.pdf}
    \parbox[b]{12.5cm} {
    The oscillator NCO\textsubscript1, can be used as part of a tune tracking
    phase locked loop. The tune phase $\phi$ is measured at the operating
    frequency $f$ and used to compute a sequence of frequency corrections
    $\delta\!f$ to maintain the phase error $\delta\phi$ at zero. The phase and
    frequency are configured via EPICS.
    }
}
\column{0.5}
    \block{Detector}{
    \includegraphics[scale=3, page=7]{figures.pdf}
    \parbox[b]{12.5cm} {
    Data from the beam is mixed with the excitation waveform in the detector to
    measure a complete complex IQ response. Operation of this system produces a
    waveform of IQ measurements which can be used for measuring betatron tune
    and other machine parameters.
    }
}
\end{columns}

\end{document}
